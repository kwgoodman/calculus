\documentclass{article}
\usepackage[left=3cm,right=3cm,top=0cm,bottom=2cm]{geometry} % page settings
\usepackage{amsmath}
\usepackage{mathtools}

\begin{document}

\title{What I think about when I think about calculus}
\author{Keith Goodman}
\maketitle

\subsection*{Introduction}

I haven't done calculus in decades. Well, that's no longer true because I
recently took a single-variable calculus course. The 13-week online class was
taught by Robert Ghrist. He's the calculus teacher you wish you had.

This document contains some of the ideas that we covered in class and some of
the ideas that occured to me while taking the class.

\subsection*{Partial sums}

After helping my daughter with arithmetic and geometric sequences and series in
Algebra 2 and Precalculus I didn't think I could get excited about finding
partial sums. Yet that's exactly what happened when we covered the subject in
class.

Our motivation in this section is to simplify the partial sum

\[
    S(k)=\sum\limits_{n=0}^k n^3-4n^2.
\]

\noindent Notice that the series is neither arithmetic nor geometric---finally
something new.

The partial sum is analogous to the definite integral $I(k)$ in continuous
calculus:

\[
I(k)=\int\limits_{n=0}^k f(n)dn=\int\limits_{n=0}^k\frac{dF(n)}{dn}dn=F(k)-F(0)
\]

\noindent where $dn$ corresponds to $\Delta n=1$ in the discrete case and
$F(n)$ is an antiderivative of $f(n)$.

We will proceed in an analogous fashion for the discrete case:

\begin{align*}
    S(k) &= \sum\limits_{n=0}^k a_n = \sum\limits_{n=0}^k (\Delta b)_n \\
         &= (b_1 - b_0)+(b_2 - b_1)+ \ldots + (b_k - b_{k-1}) + (b_{k+1} - b_k) \\
         &= b_{k+1} - b_0
\end{align*}

\noindent where $(\Delta b)_n = b_{n+1} - b_n$, the forward difference, is the
discrete derivative and $b$ is a discrete antiderivative of $a$.

The problem of simplifying $S(k)$ has now been reduced to finding a discrete
antiderivative of $n^3-4n^2$. In other words we need to find a sequence $b_n$
such that $(\Delta b)_n = n^3-4n^2$. It is easy to find a recursion relation:
$b_{n+1} = n^3-3n^2 + b_n$. However, to fully simplify $S(k)$, we need an
analytic solution for $b$, which we will obtain by using falling powers.

Falling powers are defined as

\[
    n^{\underline{k}} = n(n-1)(n-2)\ldots(n-k+2)(n-k+1)
\]

\noindent where $k$ is an integer and, by a second definition,
$n^{\underline{0}}=1$. For example

\[
    n^{\underline{3}} = n(n-1)(n-2) = n^3-3n^2+2n.
\]

\noindent Solving for $n^3$ gives $n^3 = n^{\underline{3}} + 3n^2-2n$.
Similarly $n^2 = n^{\underline{2}} + n$. After a little bit of algebra we get

\[
(\Delta b)_n=n^3-4n^2=n^{\underline{3}}-n^{\underline{2}}-3n^{\underline{1}}.
\]

The whole point of using falling powers is to (gently) coerce the problem into
a form solvable by the familiar mechanics of continuous calculus.  After three
lines of algebra, which we will skip, the discrete derivative of
$n^{\underline{k}}$ is given by the discrete power rule:
$\Delta n^{\underline{k}}= kn^{\underline{k-1}}$. Therefore $b$ is quite
simply given by

\[
b_n=\frac{1}{4}n^{\underline{4}}-\frac{1}{3}n^{\underline{3}}-\frac{3}{2}n^{\underline{2}}+C
\]

\noindent where $C$ is a constant. (To see that $b_n$ is correct we would
differentiate it to obtain
$n^{\underline{3}}-n^{\underline{2}}-3n^{\underline{1}}.$)

Putting it all together:

\begin{align*}
    S(k) &= \sum\limits_{n=0}^k n^3-4n^2 \\
         &= \sum\limits_{n=0}^k \Delta (\frac{1}{4}n^{\underline{4}}-\frac{1}{3}n^{\underline{3}}-\frac{3}{2}n^{\underline{2}}+C) \\
         &= \frac{1}{4}(k+1)^{\underline{4}}-\frac{1}{3}(k+1)^{\underline{3}}-\frac{3}{2}(k+1)^{\underline{2}} \\
         &= \frac{1}{12}k(k+1)(3k^2-13k-8)
\end{align*}

\noindent where, if we wish to use the last line, $k > 2$ lest we run into
troubles such as $3^{\underline{4}}=0$. For $2 \geq k \geq 0$, the partial sum
$S(k)$ is easily calculated by hand.

Let's try a simpler example:

\[
    \sum\limits_{n=0}^k n^2 \\
    =\sum\limits_{n=0}^k n^{\underline{2}}+n^{\underline{1}} \\
    =\sum\limits_{n=0}^k \Delta (\frac{1}{3}n^{\underline{3}}+\frac{1}{2}n^{\underline{2}}) \\
    =\frac{1}{3}(k+1)^{\underline{3}}+\frac{1}{2}(k+1)^{\underline{2}} \\
    =\frac{1}{6}k(k+1)(2k+1)
\]

\noindent where, if we wish to use the last part, $k > 1$.

Without much effort we were able to use the results we already know from
continuous calculus to find partial sums of discrete series. The techniques we
used in this section allow us to find partial sums of sequences $a_n$ that can
be written as polynomials in $n$ and some of the sequences that can be written
as rational functions of $n$.

\end{document}
